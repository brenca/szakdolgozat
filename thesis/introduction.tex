%----------------------------------------------------------------------------
\chapter*{Introduction}\addcontentsline{toc}{chapter}{Introduction}
%----------------------------------------------------------------------------

The main goal of the thesis is to create a modular interpreter in the native language of the modern Internet, JavaScript. The idea behind an interpreter such as this is that usually for someone new to programming, it is a hassle to download and install the various compilers and IDEs needed to even begin the actual programming phase, and this could be made easier with web technologies.

Nowadays, everyone has a laptop and a phone capable of browsing the web, and all these devices have a browser to do just that. Most dynamic websites use JavaScript as a language to manipulate the page (to make it dynamic), so all these browsers have a JavaScript engine built into them. I intend to use this capability to bring a more seamless first experience to those who want to learn programming and are at the beginning of their journey. 

This is why I set out to create an interpreter that is:
\begin{itemize}
\item Written in JavaScript, thus available to be used on every kind of device
\item Modular, thus making the creators of educational tools easier by abstracting the common things, and providing an easy way to allow execution of code inside the browser in any custom language that has a module created for it
\end{itemize}

To execute code written in a specific language, either a compiler (that translates directly into machine code), or an interpreter is used (which translate the code to another language higher than machine code). JavaScript is an interpreted language, partly that is why it can be found in so many places and thus the interpreter described in this thesis will be a cross-interpreter - that is, it will translate from an interpreted language to another interpreted language. But before any interpretation can begin, the code has to go through some transformations, mainly lexical analysis and parsing.

Lexical analysis is basically turning the source code into tokens using basic pattern matching rules, which will make the parsing stage easier. A token has a value and a class - the pattern matching pairs the values to their classes. A good and widely used tool for pattern matching is regular expressions, which is available in JavaScript as part of the language, and thus the proposed interpreter shall use. In case the writer of the module needs more complex rules, the lexical analyser should provide a way to use simple states and define transitions between these states.

In the next stage called parsing (or syntactical analysis), the tokens are turned into a parse tree using simple rules. A parse tree represents the hierarchy of the matched rules. One common way to define these rules is BNF (Backus?Naur form), which is a language to describe grammars. The given grammar describes every type of sentence in the language. In the definitions, we should be able to use the token classes from the previous stage (rules should be automatically added to the grammar). The parse tree then can be used to create an AST (Abstract Syntax Tree), which then can be used to interpret the code itself by evaluating the nodes, which basically represent nested parameters, some with fixed values, and some that need some computation to produce a value (and of course all these can have side effects).

To create a language module, the necessary steps are:
\begin{enumerate}
\item Create the rules for the lexical analyser
\item Provide the grammar in BNF
\item Parse the code, and iterate through the AST
\end{enumerate}
The last step is the actual interpretation, which can be a simple evaluation of an expression, or a small virtual machine that keeps track of functions, variables, scopes and etc. This is the part that requires most attention when writing a language module.

Of course, something like this has been attempted many times, there are numerous solutions to creating parsers, although most of them are not written using JavaScript, they are more native tools to generate parsers. The list includes GNU Bison \footnote{GNU Bison \url{https://www.gnu.org/software/bison/}}, Jison \footnote{Jison - Your friendly JavaScript parser generator! \url{http://zaa.ch/jison/}} (a clone of Bison in JavaScript), YACC \footnote{Yet Another Compiler-Compiler \url{http://dinosaur.compilertools.net/}} and ANTLR \footnote{Another Tool For Language Recognition \url{http://www.antlr.org/}}. But out of all of the available works, the most exciting is Syntax \footnote{Syntax - language agnostic parser generator \url{https://medium.com/@DmitrySoshnikov/syntax-language-agnostic-parser-generator-bd24468d7cfc}}, which has the same goals (parser in JavaScript) and the same things in mind (educational usage) and my thesis idea, and which was published just a week after I've handed in my thesis proposal. The author of Syntax describes his motives in the article and the main differences from my thesis are the following:
\begin{itemize}
\item He only set out to create a modular (language agnostic) parser
\item He provides multiple parsing algorithms
\item He provides multiple target languages, including JavaScript, but that only means that the parser that is generated will be a JavaScript module (other targets include Python, Ruby and PHP)
\item He provides a way to set operator precedence
\item He does not provide a state based lexical analyser
\end{itemize}
In every other sense, the two projects are very similar, although I have only learned about Syntax during later stages of development.

Another similar project is my own project - from which the idea came for this thesis -, which is a Logo interpreter that translates to JavaScript, and which was based on absolutely no research about parsers, and thus is buggy but generally works quite well. It had two iterations, the first of which I built 4 years ago as a hobby project. The second iteration is also a hobby project, but it is much more mature, and can handle Logo source pretty well. The choice of language - Logo - is because it is a good step towards understanding of basic ideas of programming (in both functional and imperative paradigms), and also provides the user with something more materialistic (it is easy to create beautiful and complex drawings), which is great for young children to get a reward for programming, which is a great way to motivate them. The most beautiful things they can draw are fractals (up to a certain level of depth), which also teach them a lot about recursion, which can be hard to understand for new pupils (fractals are a good way to visualize recursion).

Other solutions to running code on the web include cloud based solutions, which means that the code that the user inputs is actually compiled and ran on a remote virtual machine. While this is great in the sense that it is exactly as the "real thing", it has a great drawback - it requires an active internet connection to use. My solution can be bundled into a desktop application (or the website can be downloaded) using tools like the electron \footnote{Electron \url{http://electron.atom.io/}} framework. This allows the students to practice while on the go (and without an active internet connection), which cloud based solutions can not allow.

During the implementation, I will not be focusing on optimal speeds, since this is not a tool built for performance but rather one that is built to be available everywhere. To achieve modern, well readable source code and portability, I will be using the ECMAScript 6 coding standard, which is available in some modern browsers, but not so many older ones. I will develop my solution in a Node.js module, which can then be converted into code that works on older browsers using Babel \footnote{Babel \url{https://github.com/babel/babel}} and Browserify \footnote{Browserify \url{http://browserify.org/}}, which are Node.js modules to convert to older standards and to bundle modules into one source file for browsers respectively.

In Chapter 1, I will describe the implemented lexical analyser, and it's inner workings. In Chapter 2, I will write about the parser module, and the implemented parsing algorithm. In Chapter 3, I will describe the two example modules I have created - a math expression parser, and a simple Logo subset interpreter.
