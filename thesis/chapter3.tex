%----------------------------------------------------------------------------
\chapter{Parsing (Syntactical analysis)}
%----------------------------------------------------------------------------
Parsing (or syntactical analysis) is the process of turning the tokens that the lexical analysis produces into a parse tree by using a grammar. The lexical analysis phase only determines the types of things in the source code, while during parsing the higher level structures are revealed. In natural language, the lexical analysis would only reveal whether the word is a noun, verb, adverb etc., while syntactical analysis would turn these words into sentences, paragraphs, chapters and so on.

The input of the parser is the grammar and the tokens, while the output is a hierarchical structure that represents the structure of the source code (or the tokens, in this case).
%----------------------------------------------------------------------------
\section{Describing the grammar}
%----------------------------------------------------------------------------
There are well established ways to describe grammars - one of them is a notation called Backus?Naur form, or BNF. This notation is widely used, and provides an easy way to describe context-free grammars. Context-free grammars are a subset of formal grammars which do not use context to determine which rule to use.

A grammar consists of basic production rules which can be one-to-one, one-to-many or one-to-none rules, for example:
\begin{grammar}
<Sum> ::= <Sum> \lit{+} <One>
    \alt <One> 

<One> ::= \lit{1}
\end{grammar}

This grammar describes a language of only two tokens (or terminals), '+' and '1', and describes two rules:
\begin{itemize}
\item A Sum can be a Sum plus ('+') a One, or a One on it's own
\item A One is a '1' character
\end{itemize}
Or in short, an infinite amount of ones added to other ones is valid in this grammar's rule set, and the sum of the ones can be easily calculated recursively after parsing. The '|' symbol means 'or', other rules are referenced with their name written in '<>'s, and string literals are represented in quotes.

BNF has been extended over it's existence, the two notable extensions is EBNF and ABNF, where the first characters stand for extended and augmented respectively. The parser implemented in this thesis uses the original BNF syntax.
%----------------------------------------------------------------------------
\section{The parsing algorithm}
%----------------------------------------------------------------------------

%----------------------------------------------------------------------------
\section{The Parser module}
%----------------------------------------------------------------------------



